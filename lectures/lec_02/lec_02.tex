\title{Multicore Computing Lecture 2}

\author{
        Chunheng Luo \\
}

\date{\today}

\documentclass[12pt]{article}
\setlength{\parindent}{0em} 
\setlength{\parskip}{1em}
\renewcommand{\baselinestretch}{1.0} 
\usepackage{listings}
\usepackage{color}

\begin{document}
\maketitle

\section{Implementing Threads in Java}
\begin{enumerate}
\item Extend the Thread interface $\rightarrow$ start

\item Implement Runnable/Callable $\rightarrow$ create threads $\rightarrow$ start

[Example: FooBar.java]

\item Executor services 


\textbf{Synchronous vs. asynchronous multithreading } 
When you execute something synchronously, you wait for it to finish before
moving on to another task. When you execute something asynchronously, you can
move on to another task before it finishes. 

\begin{itemize}
\item Synchronous execution 

\textit{Using the ``join'' method: blocking construct} \\
Wait for all running threads to finish using join, and then start new threads. 

[Example: Fibonacci.java]

\item Asynchronous execution 

\begin{itemize}
\item Using ExecutorService and ThreadPool \\
Submit threads to a thread pool, which manages the execution of all threads.
Asynchronous execution is supported. 
Each submitted thread returns an object of ``future'' type, which represents the output of the
thread that has not been available at the moment. The ``get'' method of the
``future'' method waits until the thread finishes and then retrieves the final
result. 

[Example: Fibonacci2.java] \\

\item Using RecursiveTask \textit{to ``fork'' and ``join'' threads} \\
The method ``fork'' arranges to asynchronously execute the task in the pool
the current task is running in. 
[Example: Fibonacci3.java]  \\
\begin{lstlisting}[language=Java]
  protected Integer compute() {
    if ((n == 0)||(n == 1 )) return 1;
    Fibonacci3 f1 = new Fibonacci3(n - 1);
    f1.fork();
    Fibonacci3 f2 = new Fibonacci3(n - 2);
    return f2.compute() + f1.join();
  }
  /* 
   * By directly compute f2 first, the result of 
   * Fibonacci3(n-2) is reused in the computation 
   * of Fibonacci3(n-1), which is done during f1.join()
  */
\end{lstlisting} 
\end{itemize}
\end{itemize}
\end{enumerate} 

\section{Amdahl's Law} 
$P$: fraction of the work that can be done in parallel \\
$n$: number of cores \\
$T_s$: time on sequential processor \\
$T_p$: time on multicore machine \\

$$T_p\geq(1-P)T_s+\frac{PT_s}{n}$$ 

The speedup is 
$$\frac{T_s}{T_p}\leq\frac{1}{1-P+\frac{P}{n}}$$

\section{Mutual Exclusion} 
If any core is in a crital section (CS), all other cores should not be in the
critical section 

Example: two cores are both doing x:=x+1, they cannot do it at the same time.
(Wrong values can be wirtten into the mem otherwise. ) 

\textbf{Some tentative solutions: } 
\begin{itemize} 
\item Attemp1.java  \\ 
Several threads can ``lock the door'' simultaneously, which violates the mutual
exclusion requirement.  
\item Attemp2.java \\
If all threads request CS simultaneously, none will enter CS. That is called a
deadlock. 
\item Attemp3.java \\ 
If all threads assign the variable ``turn'' simultaneously, a deadlock occurs. 
\end{itemize} 

\textbf{Canonical solution: Peterson Algorithm }
\begin{lstlisting}[language=C]
  /* For core P0 -------------------------------------- */
  wantCS[0] = true;  // Raise the request to enter CS
  turn = 1;          // Let the other core go first 
  while ( wantCS[1] && turn == 1 ) NO_OP();  
  {CS} 
  wantCS[0] = false; 
  
  /* For core P1 -------------------------------------- */
  wantCS[1] = true;  // Raise the request to enter CS
  turn = 0;          // Let the other core go first 
  while ( wantCS[0] && turn == 0 ) NO_OP();  
  {CS} 
  wantCS[1] = false; 
\end{lstlisting} 

\textbf{Proof: }

\begin{enumerate}
\item Deadlock-free 

Deadlock \\ 
$\Rightarrow$ ( wantCS[1] \&\& turn == 1 ) \&\& ( wantCS[0] \&\& turn == 0 ) \\
But turn cannot be 1 and 0 at the same time $\Rightarrow$ conflict 

\item Mutual exclusion 

Assume both are in CS, and let the value of turn be 1.  \\ 
Then at some point before this, turn = 0. \\ 
Then at some point further before, wantCS[1] = true happens. \\
Now we have ( wantCS[1] \&\& turn == 1 ) $\Rightarrow$ P0 cannot enter CS. \\
\\
\textit{This informal proof does not seem to be correct. The formal proof
(Dijkstra's proof) should be
in the course material, and can be found on the Internet.  }
\end{enumerate}

\bibliographystyle{abbrv}
\bibliography{simple}

\end{document}
