%
% This is the LaTeX template file for lecture notes for EE 382C/EE 361C.
%
% To familiarize yourself with this template, the body contains
% some examples of its use.  Look them over.  Then you can
% run LaTeX on this file.  After you have LaTeXed this file then
% you can look over the result either by printing it out with
% dvips or using xdvi.
%
% This template is based on the template for Prof. Sinclair's CS 270.

\documentclass[twoside]{article}
\usepackage{graphics}
\setlength{\oddsidemargin}{0.25 in}
\setlength{\evensidemargin}{-0.25 in}
\setlength{\topmargin}{-0.6 in}
\setlength{\textwidth}{6.5 in}
\setlength{\textheight}{8.5 in}
\setlength{\headsep}{0.75 in}
\setlength{\parindent}{0 in}
\setlength{\parskip}{0.1 in}

%
% The following commands set up the lecnum (lecture number)
% counter and make various numbering schemes work relative
% to the lecture number.
%
\newcounter{lecnum}
\renewcommand{\thepage}{\thelecnum-\arabic{page}}
\renewcommand{\thesection}{\thelecnum.\arabic{section}}
\renewcommand{\theequation}{\thelecnum.\arabic{equation}}
\renewcommand{\thefigure}{\thelecnum.\arabic{figure}}
\renewcommand{\thetable}{\thelecnum.\arabic{table}}

%
% The following macro is used to generate the header.
%
\newcommand{\lecture}[4]{
   \pagestyle{myheadings}
   \thispagestyle{plain}
   \newpage
   \setcounter{lecnum}{#1}
   \setcounter{page}{1}
   \noindent
   \begin{center}
   \framebox{
      \vbox{\vspace{2mm}
    \hbox to 6.28in { {\bf EE 382C/361C: Multicore Computing
                        \hfill Fall 2016} }
       \vspace{4mm}
       \hbox to 6.28in { {\Large \hfill Lecture #1: #2  \hfill} }
       \vspace{2mm}
       \hbox to 6.28in { {\it Lecturer: #3 \hfill Scribe: #4} }
      \vspace{2mm}}
   }
   \end{center}
   \markboth{Lecture #1: #2}{Lecture #1: #2}
   %{\bf Disclaimer}: {\it These notes have not been subjected to the
   %usual scrutiny reserved for formal publications.  They may be distributed
   %outside this class only with the permission of the Instructor.}
   \vspace*{4mm}
}

%
% Convention for citations is authors' initials followed by the year.
% For example, to cite a paper by Leighton and Maggs you would type
% \cite{LM89}, and to cite a paper by Strassen you would type \cite{S69}.
% (To avoid bibliography problems, for now we redefine the \cite command.)
% Also commands that create a suitable format for the reference list.
\renewcommand{\cite}[1]{[#1]}
\def\beginrefs{\begin{list}%
        {[\arabic{equation}]}{\usecounter{equation}
         \setlength{\leftmargin}{2.0truecm}\setlength{\labelsep}{0.4truecm}%
         \setlength{\labelwidth}{1.6truecm}}}
\def\endrefs{\end{list}}
\def\bibentry#1{\item[\hbox{[#1]}]}

%Use this command for a figure; it puts a figure in wherever you want it.
%usage: \fig{NUMBER}{SPACE-IN-INCHES}{CAPTION}
\newcommand{\fig}[3]{
			\vspace{#2}
			\begin{center}
			Figure \thelecnum.#1:~#3
			\end{center}
	}
% Use these for theorems, lemmas, proofs, etc.
\newtheorem{theorem}{Theorem}[lecnum]
\newtheorem{lemma}[theorem]{Lemma}
\newtheorem{proposition}[theorem]{Proposition}
\newtheorem{claim}[theorem]{Claim}
\newtheorem{corollary}[theorem]{Corollary}
\newtheorem{definition}[theorem]{Definition}
\newenvironment{proof}{{\bf Proof:}}{\hfill\rule{2mm}{2mm}}

% **** IF YOU WANT TO DEFINE ADDITIONAL MACROS FOR YOURSELF, PUT THEM HERE:

\begin{document}
%FILL IN THE RIGHT INFO.
%\lecture{**LECTURE-NUMBER**}{**DATE**}{**LECTURER**}{**SCRIBE**}
\lecture{27}{November 29}{Vijay Garg}{Chunheng Luo}
%\footnotetext{These notes are partially based on those of Nigel Mansell.}

% **** YOUR NOTES GO HERE:

% Some general latex examples and examples making use of the
% macros follow.  
%**** IN GENERAL, BE BRIEF. LONG SCRIBE NOTES, NO MATTER HOW WELL WRITTEN,
%**** ARE NEVER READ BY ANYBODY.
\section*{Question 1}
Which of the following statements about blocks and threads in CUDA are \textbf{FALSE}? \\ \\
\textbf{A}. The maximum number of threads per block is determined by the hardware architecture of the GPU. \\
\textbf{B}. The method {\em \_\_syncthreads()} acts as a barrier to synchronize all threads within a block. \\
\textbf{C}. Two threads located in different blocks can synchronize with each other within the kernel. \\
\textbf{D}. Branches (e.g. if/else) in the kernel can potentially reduce parallelism between threads. \\ \\ 
Solution: C

\section*{Question 2} 
Which of the following statements about shared and global memory in CUDA are \textbf{FALSE}? \\ \\
\textbf{A}. Global memory has the scope of the entire application. \\
\textbf{B}. Shared memory is usually on-chip and faster than global memory. \\
\textbf{C}. Global memory has to be declared outside the kernel. \\
\textbf{D}. Shared memory cannot be accessed by the host. \\ \\
Solution: C

\section*{Question 3} 
Which of the following statements about kernel functions in CUDA are \textbf{TRUE}? \\ \\
\textbf{A}. Kernel functions are called by the host and run on the device. \\
\textbf{B}. The return type of kernel functions can only be {\em void} \\
\textbf{C}. We can have multiple kernel invocations in an application. \\
\textbf{D}. All of the above. \\ \\ 
Solution: D

\section*{Question 4}
In the work-optimal algorithm for computing array maximum discussed in class, \\ \\
\textbf{A}. the array of size {\em n} is partitioned into {\em (n/log log n)} groups of size {\em O(log log n)}. \\
\textbf{B}. sequential algorithm is run on each of the groups to get the maximum value of each group. \\
\textbf{C}. the doubly logarithmic height tree algorithm is used to get the maximum value of the group maximums. \\
\textbf{D}. All of the above. \\ \\ 
Solution: D

\section*{Question 5}
In the work-optimal algorithm for merging two sorted arrays discussed in class, \\ \\ 
\textbf{A}. the ranks of splitters are computed using the parallel rank-computation algorithm. \\
\textbf{B}. sublists are merged using the sequential merging algorithm. \\
\textbf{C}. the overall time complexity is {\em O(log n)}. \\
\textbf{D}. All of the above. \\ \\ 
Solution: D

\section*{Question 6}
In the list ranking algorithm discussed in class, what technique is used to reduce the length of the longest path in the linked structure in each iteration? \\ \\ 
Solution: Pointer jumping

\section*{Question 7}
In the breaking symmetry algorithm discussed in class, what is a possible method to guarantee that initially all nodes have valid coloring? \\ \\
Solution: Each node could initially have different color by using its unique identifier. 

\section*{Question 8}
Briefly describe the upward sweep and downward sweep procedures in the Blelloch Scan algorithm. \\ \\ 
Solution: \\
\textit{Upward}: 
{\em sum[v] = sum[Left[v]] + sum[Right[v]]; } \\
\textit{Downward}:  
{\em scan[Left[v]] = scan[v]; scan[Right[v]] = sum[Left[v]] + scan[v]; }

\end{document}





